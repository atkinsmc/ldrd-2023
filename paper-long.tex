\documentclass[12pt]{article}
\usepackage[hmargin=1in,vmargin=1in,includefoot]{geometry} 
\usepackage{braket}
\usepackage{pgfgantt}
\usepackage{tikz}
\usepackage{relsize}
\usepackage{graphicx}% Include figure files
\usepackage{enumitem} %enumerate with letters, etc
\usepackage{float} %to better place figures
\usepackage{amsmath}
\usepackage{amsthm}
\usepackage{amssymb}
\usepackage{bm}
\usepackage{wrapfig}
%\usepackage{setspace}
%\doublespacing

\usetikzlibrary{tikzmark,fit,shapes.geometric}
\usetikzlibrary{arrows,shapes,positioning}
\usetikzlibrary{decorations.markings}

\definecolor{darkred}{RGB}{128,0,0}
\definecolor{palegoldenrod}{RGB}{238,232,170}

\tikzstyle arrowstyle=[scale=1]

\tikzstyle directed=[postaction={decorate,decoration={markings,mark=at position .5 with {\arrow[arrowstyle]{stealth}}}}]
\tikzstyle reverse directed=[postaction={decorate,decoration={markings, mark=at position .5 with {\arrowreversed[arrowstyle]{stealth};}}}]

\usepackage{mathtools}
\mathtoolsset{showonlyrefs}
%\pagenumbering{gobble}

\makeatletter
\def\@maketitle{%
  \newpage
  \vspace*{-\topskip}      % remove the initial space
  \begingroup\centering    % instead of \begin{center}
  \let \footnote \thanks
  \hrule height \z@        % to avoid the insertion of lineskip glue
    {\LARGE \@title \par}%
    \vskip 1.5em
    {\large
      \lineskip .5em
      \begin{tabular}[t]{c}%
        \@author
      \end{tabular}\par}%
  \par\endgroup            % instead of \end{center}
  \vskip 1.5em             % <--- modify this to adjust the separation
}
\makeatother

\begin{document}

\title{Forging new pathways to improved nuclear-reaction predictions}
\author{Mack C. Atkinson}
\maketitle

%\section{Introduction}
%The origin and abundance of all atomic nuclei lies at the heart of nuclear physics.
\textcolor{blue}{\textbf{Introduction}}
 - How do neutrons and protons arrange themselves to produce an atomic nucleus? This fundamental question has been the underlying motivation in the advancement of nuclear physics
 for over half a century now. After all this time, even with all the discoveries and advancements in the field, we still don't have the answer to this simple question. Much
 progress has been made toward an answer by investigating the nature of atomic nuclei through nuclear scattering experiments. By observing (and quantifying) the collision of
 nuclei, the mechanisms that dictate the subsequent motion of nucleons (protons or neutrons) can be unravelled. 
%The way in which nuclei respond in nuclear reactions helps to inform the nature of the nuclear force - the underlying interaction between neutrons and protons that governs the
%formation of nuclei. 
 To reach a fundamental understanding of nuclei, a suitable reaction theory is vital in order to interperet experimental data. Experimental techniques, instruments, and facilities
 are advancing at a rapid rate, so nuclear reaction theory must progress in tandem. A looming challenge in nuclear reaction theory is to provide state-of-the-art predictions for
 the wealth of experimental data now being produced at DOE's flagship facility for nuclear science, the Facility for Rare Isotope Beams (FRIB), which started user operations in May
 2022. FRIB is a facility that generates high-energy rare (unstable, exotic) isotope beams which allows for the study of these short-lived nuclei
 which otherwise could not be studied (since they are too short lived to act as targets). This exploration of nuclei approaching the limits of
 stability will provide new insights into the fundamental nuclear force as well as reveal emergent phenomena in the structure of these exotic systems. 
   Indeed, the evolution of nuclear structure as we approach the limits of stability is an overarching theme of the recent DOE Town Hall Meeting
 which will determine the 2023 Nuclear Long Range Plan.
 %In the 2015 Long Range Plan,
 %it was said that "Scientific discoveries are often made by investigating nature at the extremes", which indeed explains the excitement of the nuclear community for FRIB science. 
   Indeed, from the 2012 Decadal study of Nuclear Physics by the National Academies of Sience, "Nuclear Physics: Exploring the
 Heart of Matter", it was said that, "Many of today’s most important advancements in medicine, materials, energy, security, climatology, and dozens of other sciences emanate from
 the wellspring of basic research and development in nuclear physics. Answers to some of the most important questions facing our planet will come from nuclear science,
 interdisciplinary efforts in energy and climate, and marketplace innovations". It is these observations that motivate our proposal to develop more complete reaction descriptions that will help to propel our understanding of nuclear physics to the limits of stability.  
 %interdisciplinary efforts in energy and climate, and marketplace innovations". It is these observations that motivate our pursuit of more complete reaction descriptions that will
 %help to propel our understanding of nuclear physics to the limits of stability.  
 %Our proposal is designed to aid in this pursuit contribute to the scientific community's investigations into nuclear structure by developing better predictions of nuclear reactions so that data from
 %advanced experimental facilities can be fully utilized.
%Thus, the main drive of this proposal is to develop a better pathway toward describing nuclear reactions so that the experimental data coming from these billion-dollar
 %facilities can be optimally utilized.

The first big step toward understanding the structure of atomic nuclei (that is, the way in which neutrons and protons arrange themselves inside the nucleus) was the advent of the
nuclear shell model by Maria Goeppert Mayer in 1948 [CITE]. While this elegant model does a remarkable job describing stable nuclei, it diverges from reality when considering
nuclei away from stability (nuclei with excess neutrons or excess protons). Indeed, a rich set of interesting features
start to appear for these exotic nuclei (such as halo clustering, shape coexistence, deformations) that the independent shell model has no way of describing. Thus, more complicated
nuclear structure models have been developed to try and account for these features, but it is still quite difficult to correctly describe these nuclei near the limits of stability.
Naturally, these exotic phenonema can be investigated through the scattering of rare isotopes (of which many systems were not possible to study before FRIB). However, connecting
this rich nuclear structure to the experimental data is no easy task. In order to bridge the gap between scattering data and nuclear structure, nuclear reaction theories have been
developed. Thus, the path toward understanding nuclei at the
limits of stability is the unified, simultaneous development and communication between three areas: improved experimental techniques and facilities to measure unstable nuclei (i.e.
FRIB), the development of more precise nuclear structure models, and the development of better reaction theories. 
%I have identified
%an area where nuclear reaction theory can be improved, and thus this proposal is to improve reaction descriptions such that nuclear structure can more reliably be extracted from
%nuclear scattering data.

We propose to develop an effective interaction that can be used in nuclear reaction calculations. One particularly useful reaction for probing nuclear structure is
the so-called knockout reaction where a projectile is used to knockout a proton or a neutron from a nucleus. A specific case of this is an electron-induced reaction on $^{40}$Ca
where a high-energy electron beam is shot at a Calcium target resulting in the removal of a proton. The PI developed a dispersive optical model (DOM) during his PhD which provides
a consistent description of both structure and scattering states of nuclei. In particular, the PI applied the DOM to 40Ca which allowed a consistent reaction-theory description to
the electron-induced knockout $(e,e'p)$. For the first time, all ingredients needed for this DWIA description were provided from the same model resulting in a beautiful description
of the experimental data. The interesting quantity coming from this reaction is the so-called spectroscopic factor, which is like a probability for the struck proton to have been
found in the particular state it was. Now, electron scattering is a very clean way of probing atomic nuclei since the electron-proton interaction is well understand. However,
electron scattering requires a stable target - thus only stable nuclei can be investigated with it. Thus, if one wants to investigate nuclei near the limits of stability, inverse
kinematics must be utilized as is the case at FRIB. The reactions are more complicated now since they are hadron-induced, thus the nuclear force is used between the projectile and
target system. In order to investigate these hadron-induced reactions, I calculated the same proton removal from 40Ca, expect instead of being electron-induced it is proton induced
(note that this experiment could be done in inverse kinematics since a Hydrogen gas is a suitable target for a beam of unstable nuclei to strike). Once again, since the DOM is now
developed I have a consistent description of all aspects of this removal reaction.
\begin{wrapfigure}{r}{0.5\textwidth}
   \label{fig:eep_p2p}
   \begin{center}
      \includegraphics[width=0.49\linewidth]{figures/eep100.pdf}
      \includegraphics[width=0.49\linewidth]{figures/p2p.pdf}
      \includegraphics[width=0.49\linewidth]{figures/eep-schematic-alpha.png}
      \includegraphics[width=0.49\linewidth]{figures/p2p-schematic-alpha_oldV.png}
   \end{center}
   \caption{The left figure shows the results of a DWIA calculation of $^{40}$Ca$(e,e'p)^{39}$K using DOM ingredients. The right figure shows a similar DWIA calculation, except the
   probe is now a proton rather than an electron, $^{40}$Ca$(p,2p)^{39}$K. While the same DOM ingredients are used in both cases, the electron-induced knockout reaction [left] reproduces
the experimental data much more accurately than the proton-induced knockout reaction [right]. Below each curve is a schematic of each process.}
\end{wrapfigure}
However, comparing the resulting cross section with the experimental data showed a 20\% lack of strength. The only difference between the electron-induced reaction and the
proton-induced reactions is the fact that the interaction between the probe and the nucleus is now the nuclear force rather than the electromagnetic force (of course the nuclear
structure information of 40Ca is an intrinsic property so this remains unchanged). Since the DOM is uniquely suited for both of these reactions, it is now clear that the
discrepancy is coming from the proton-proton interaction between the proton probe and the struck proton in Calcium. Thus, I have an ideal testing stage for developing a better
proton-proton interaction to improve the reaction description of hadron-induced knockout. Using the DOM, I can dress the bare proton-proton interaction with the structure of 40Ca
leading to a more realistic description of the (p,2p) interaction. \textcolor{blue}{[Here I am trying to indicate what else is out there, and also that the stage is uniquely set for us to calculate
this now].} Until the DOM was developed, previous calculations of (p,2p) were always scaled to match the experimental data -
meaning that there was no way to tell if there were a discrepancy in the structure description. Because of this, the proton-proton interaction used in the (p,2p) reaction has not
seeen development for over two decades. The most recent development was in 2000 where a group from Melbourne used infinite nuclear matter (an idealized nuclear system) in order to
include some kind of nuclear-medium effects. This proposal will be the first to actually include affects from finite nuclei in the effective interaction implemented in the (p,2p)
reaction.  This perscription can then be used to improve any hadron-induced reaction, since this effective interaction will
be relevant. Since FRIB is utilizing inverse kinematics, all reactions from FRIB will be hadron-induced, thus an effective iteraction will improve the ability to extract
information from FRIB. This will then lead to the ability to properly extract nuclear structure information for nuclei reaching the limits of stability (the effect could be as much
as 20\% as we have already seen in the case of p2p). 

\textcolor{blue}{\textbf{Project Plan}}
To develop a finite-nucleus-informed effective interaction for nuclear reactions, one needs to consider the many ways in which a nucleon interacts as it propagates through a
nucleus. A natural language to discuss this is the through Green's functions (or single-particle propagators) and perturbation theory~\cite{Exposed!}. Indeed, knowing the
single-particle propagator of a nucleus is a powerful lever-arm in perturbation theory. We already have an accurate single-particle propagator from the DOM that I developed in my
PhD. With this, we have the basic building-block to construct the effective interaction. By combining a bare nucleon-nucleon interaction [probably need to say more about this???]
and the DOM single-particle propagator, the effective interaction can be calculated using already-derived many-body formalism.   
   The approximation that we can implement will be valid at energies approximately greater than 70 MeV/u, which also corresponds to the region of validity for the DWIA reaction
   description of (p,2p).  

\begin{figure}[h]
   \begin{tikzpicture}[node distance = 0.5cm]
      \node[fill=green,dashed,rectangle,thick,draw,opacity=0.2,text opacity = 1] (a) at (0.2,0) 
      {\begin{tikzpicture}
         \node[align=center,ellipse,thick,draw,solid] (b) at (-1.6,0.5) { DOM \\ \includegraphics[scale=0.15]{figures/nucleus.png}};
         \node[fill=orange,rectangle,thick,draw,solid,minimum width=0.15\linewidth] (c) at (1.7,0.5) {\small Old V};
         \node[ellipse,thick,draw,solid] (d) at (0,-1) {\small Formalism};
         \draw[solid] (b) -- (c);
         \draw[solid] (c) -- (d);
         \draw[solid] (b) -- (d);
      \end{tikzpicture}
   };
   \node[rectangle,left=of a] (a2) {=};
   \node[fill=green,rectangle,thick,draw,minimum width=0.15\linewidth,left=of a2] (a3) {\small New V};
   \node[rectangle,right=of a] (b1) {$\implies$};
   \node[rectangle] (b2) at (7,0) {\includegraphics[scale=0.25]{figures/p2p-schematic-alpha_newV.png}};
         %\node[rectangle,right=of b1] (b2) at (1,0) {\includegraphics[scale=0.25]{figures/p2p-schematic-alpha_newV.png}};
\end{tikzpicture}
   \end{figure}
   
While the many-body formalism of this approximation to an effective interaction has been known for some time now, it has only been possible to implement in the fictional system of
infinite nuclear matter [CITE]. The limitation, until now, has been obtaining an accurate single-particle propagator for finite nuclei. Now that the DOM provides the
single-particle propagator in finite nuclei, this formalism can now be applied in a realistic way. Implementing the many-body formalism in a finite system is still nontrivial
however, even starting from an already-known propagator. On a technical level, the two-particle propagator must be generated from the single-particle propagator which itself
requires approximation. Once a suitable approximation to the two-particle propagator is determined, the solution to the many-body equation to determine the effective interaction
will involve a large matrix inversion. Not only will the matrix inversion be computationally costly, but the generation of the matrix involves multi-dimensional integrals which
will be computationally demanding. The limitation of computational power has also been a contribution to the inability to implement this formalism in the past. While this many-body
formalism will be implemented in a general way, we will specifically be calculating the effective interaction in $^{40}$Ca as it is a doubly-magic nucleus and the DOM describes it
well. Generally, our objectives are to generate this effective interaction then demonstrate how it enhances reaction theory through some specific cases:
\\
\textbf{Objective I: Calculate the effective interaction in $^{40}$Ca}
I will combine a chiral NN interaction (the ones typically used in \textit{ab initio} calculations [CITE]) with the DOM proton propagator of $^{40}$Ca to generate an effective
proton-proton interaction in $^{40}$Ca. This objective will involve heavy code development as the many-body formalism has not been implemented for finite nuclei. I plan to utilize
parallel programming so that the multi-dimensional integrals needed in the many-body calculation will be tractable. I expect this objective to be the bulk of the work. The PI will
be developing the code for this objective with general guidance from Co-Is Sofia Quaglioni and Gregory Potel. The expertise of Wim Dickhoff will be invaluable when implementing
this many-body formalism. I can test my code by using it to calculate free NN scattering, since the formalism will exactly correspond to free-nucleon scattering if I use free
propagators in the calculation rather than the DOM $^{40}$Ca propagators.
\\
\textbf{Objective II: Calculate $^{40}$Ca$(p,2p)^{39}$K using the $^{40}$Ca effective interaction}
Once the effective proton-proton interaction is generated in $^{40}$Ca, it will be straightforward to calculate the $^{40}$Ca$(p,2p)^{39}$K cross section in the same was that
Fig.~\ref{fig:eep_p2p} was generated. The pipeline with outside collaborator K. Yoshida from the JAEA is already in place to use the DOM ingredients in the reaction calculation.
The only difference in the calculation will be that now I can also provide the proton-proton interaction that will be used in the calculation. We don't foresee any complications
arising in linking our new effective interaction with the existing $(p,2p)$ reaction code. This will result in an improved calculation which will ideally address the discrepancy
see in Fig.~\ref{fig:eep_p2p}. This will confirm that the effective interaction is indeed incorporating in-medium effects in a realistic way. At this point, the machinery to
generate the effective interaction is in place and has been demonstrated to be working. While the improvement of $(p,2p)$ is itself and accomplishment, there are many more
applications of this effective interaction, as was indicated in the Introduction.
\\
\textbf{Objective III: From the effective interaction, calculate deuteron scattering in $^{40}$Ca}
After demonstrating the usefulness of the effective interaction in proton-induced knockout, we can also use it to improve deutron-based reactions. The deuteron is is a nucleus
consisting of one proton and one neutron bound together. Thus, the relevant interaction involving deuterons is a proton-neutron interaction. Instead of calculating the
proton-proton effective interaction, it is no extra work to calculate the proton-neutron effective interaction. Using this proton-neutron effective interaction, we can calculate
the proton-neutron propagation through a nucleus (in this specific case it will be $^{40}$Ca). Using this proton-neutron propagator, we can describe deuteron elastic scattering
with $^{40}$Ca. Once this is established and verified by comparing with elastic scattering data, the deuteron propagator can then be used to improve more complicated reactions such
as $(d,p)$ and $(p,d)$. These reactions can act as surrogates to neutron-induced reactions (neutron capture, for example). This objective will demonstrate another class of
reactions that will benefit from the calculation of this effective interaction. The PI will work closely with co-PI Gregory Potel, and expert in deuteron scattering, when
implementing the deuteron propagator.
\\
\textbf{Objective IV: From the effective interaction, calculate an optical potential in $^{40}$Ca}
Another immediate application our effective interaction is the ability to generate an proton (or neutron) optical potential. An optical potential describes the scattering of
protons (or neutrons) with nuclei (in this case it will be $^{40}$Ca). Almost any theoretical description of a reaction involving medium-to-heavy mass nuclei requires an optical
potential. The DOM is a purely phenomenological potential. Building an optical potential from our effective interaction is a way of explicitly including the $NN$ interaction. Of
course, we are using DOM propagators to generate the effective interaction so the resulting optical potential will still be phenomenological by construction, however it will now be
an optical potential that has microscopic features. This provides a handle on how the $NN$ interaction affects elastic-scattering observables. We see two motivations for generating
this optical potential. The first is that it provides an improvement (in the form of microscopic elements) to the DOM, marking  the advent of a new class of hybrid optical
potentials. The PI will work closely with co-PI C. Pruitt, who is also an expert on the DOM, to incorporate these microscopic elements to enhance the DOM. The second is that it paves the way toward a truly \textit{ab initio} optical potential. We will have a method that generates an optical potential from the NN
interaction if the propagator is known. Thus, if one were to use a propagator generated from a simpler approximation, say at the Hartree-Fock level, then our machinery provides a
systematic way of improving this optical potential. The ability to construct an optical potential microscopically is highly attractive since it provides a controlled way of
predicting the scattering of exotic nuclei that cannot possibly be extrapolated from phenomenological optical potentials. 
\\
\textbf{Deliverables and Milestones}
The timeline of the project is outlined in the Gantt chart below. Deliverables consist of publishing three papers in high-profiles papers, one for each implementation of the
developed interaction.
\begin{figure}[h]
\begin{center}

%\begin{ganttchart}[y unit title=0.4cm,
%y unit chart=0.5cm,
%title label anchor/.style={below=-1.6ex},
%title height=1,
%progress label text={},
%bar height=0.7,
%group right shift=0,
%group top shift=.6,
%group height=.3]{1}{8}
   \begin{ganttchart}[bar/.append style={fill=red!50},vgrid,hgrid,
      y unit chart=0.6cm,
title label anchor/.style={below=-1.6ex},
title height=1,
y unit title=0.5cm,
Mile1/.style={milestone/.append style={fill=red}},
  Mile2/.style={milestone/.append style={fill=blue,shape=rectangle}}
   ]{1}{8}
\gantttitle{Time (in Quarters)}{8} \\
\gantttitle{1}{1}
\gantttitle{2}{1}
\gantttitle{3}{1}
\gantttitle{4}{1}
\gantttitle{5}{1}
\gantttitle{6}{1}
\gantttitle{7}{1}
\gantttitle{8}{1}\\
%tasks
\ganttbar{Task: evaluate $\Gamma$}{1}{4} \\
\ganttmilestone[Mile1]{Milestone: \textnormal{$\Gamma$ calculated}}{4}\\
\ganttbar{Task: Implement $\Gamma$ in $(p,2p)$}{4}{5} \\
\ganttmilestone[Mile2]{Deliverable: \textnormal{Publication on improved $(p,2p)$ description}}{5}{5} \\
\ganttbar{Task: Calculate deuteron scattering in 40Ca}{5}{7} \\
\ganttmilestone[Mile2]{Deliverable: \textnormal{Publication on method deuteron scattering}}{7}{7} \\
\ganttbar{Task: Generate optical potential}{7}{8} \\
\ganttmilestone[Mile2]{Deliverable: \textnormal{Publication on hybrid optical potential}}{8}{8}


%relations
\ganttlink{elem0}{elem1}
\ganttlink{elem2}{elem3}
\ganttlink{elem4}{elem5}
\ganttlink{elem6}{elem7}
%\ganttlink{elem1}{elem2}
%\ganttlink{elem3}{elem4}
%\ganttlink{elem1}{elem5}
%\ganttlink{elem3}{elem5}
%\ganttlink{elem2}{elem6}
%\ganttlink{elem3}{elem6}
%\ganttlink{elem5}{elem7}
\end{ganttchart}
\end{center}
\end{figure}
\\
\textbf{Project Impact}
This project builds a pathway toward better reaction predictions. This work will enhance the lab's capability to contribute to the understanding of the data soon to come from FRIB.
This project will boost LLNL's capability to calculate complex nuclear reactions. The improved reaction descriptions from this program  will allow LLNL to stay at the forefront for
probing new physics in exotic nuclei. Furthermore, it will enhance the lab's Stockpile Stewardship mission by providing better calculations of relevant reactions such as neutron
capture. It will also helpt to cultivate a good relationship with FRIB since it will lead to better calculations of FRIB-relevant reactions. It will help strengthen connections to
institutions involved in the project, Washington University in St. Louis as well as the Japan Atomic Energy Agency. 
\\
\textbf{Risks and Mitigations}
Even with the reaction description improved dramatically (and made more consistent), it is possible that implementing the effective interaction won't
fully account for the discrepancy in the (p,2p) results in Fig.~\ref{fig:eep_p2p}. If this turns out to be the case, the work is still important because it will inform us that
something in the reaction description itself is now the problem. With a full treatment of the boundstates, scattering states, and effective proton-proton interaction consistently
derived from the same DOM potential, this calculation would show without ambiguity that the reaction description needs more development. So, even if the effective interaction does
not fully solve the discrepancy, this project will still lead to improved reaction descriptions. Furthermore, the improvement of the particular (p,2p) reaction is only one
objective in this proposal. The effective interaction opens doors to improvements of any hadron-induced reactions, so we will also explore the effect it has on deuteron scattering
and phenomenological optical potentials.
\\
\textbf{Project Team} The team consists of postdoctoarl researcher Mack Atkinson (PI), staff scientist Gregory Potel (co-I), staff scientist Cole Pruitt (co-I), and NACS group
leader Sofia Quaglioni (co-I), outside collaborator Willem Dickhoff (full professor at Washington University in St. Louis), and outside collaborator Yoshida Kazuki (staff
scientist at the Japan Atomic Energy Agency (JAEA)). Atkinson is an expert in many-body theory and reaction theory. He will be developing the code to calculate the effective interaction. Potel is an
expert in reaction theory, and particularly on deuteron-based reactions. He will be involved with Objective III when implementing the effective interaction to calculate deuteron
scattering. Pruitt is an expert in optical potentials and the DOM. He will assist in integrating of the effective-interaction-derived optical potential into the DOM to go towards
hybrid optical potentials. Quaglioni is a world-recognized expert in \textit{ab initio} reaction theory who will provide guidance throughout the life of the project. Dickhoff is an
expert in many-body theory, he will lend support for objective I when calculating the effective interaction in 40Ca. Yoshida is an expert in (p,2p) reactions and will be heavily
involved in objective II when the effective interaction is applied in the $^{40}$Ca$(p,2p)^{39}$K reaction.
\\
\textbf{Budget}
 - We request funding in the amount of \$221k for Year 1 and \$230k for Year 2 to support: the PI at 50\%, Potel at 10\%, Pruitt at 10\%, and Quaglioni at 5\%. The funding will
 also include travel for the PI to disseminate the results of the project at conferences/workshops.
\\
\textbf{Exit Plan}
This project will enhance LLNL's capability to predict a variety of high-energy nuclear reactions important for nuclear astrophysics and National Security applications which will
help to grow the LLNL's DOE Office of Science (DOE/SC) base funding for nuclear physics. This project will boost the visibility of the PI and set the stage for pursuing a DOE Early
Career Award. Upon completion of the effective interaction, the applications to other nuclear reactions are many. While this project focuses on $^{40}$Ca the code to generate the
effective interaction will be agnostic to the particular nucleus, so this method can then be applied to any nucleus. Thus, we can continue to utilize the machinery for generating
the effective interactions relevant to many other reactions. Furthermore, the particular reactions that we choose to demonstrate during the project are just a subset of what is
possible with this formalism. Transfer reactions, stripping reactions, (p,pn) are some examples of interesting reactions to study with this effective interaction. The consistent
description of the (p,2p) reaction can lead to an analysis of many other (p,2p) data and could even help to address the polarization puzzle to which there is no current solution.
With the endorsement of FRIB400 (a higher energy beam at FRIB) in the Nuclear Structure and Reactions Town Hall Meeting in 2022, there will certainly be many new (p,2p) experiments
being performed at the limits of stability whose analysis will benefit from this effective interaction machinery. The deuteron studies will be highly relevant for many
deuteron-based experiments at FRIB. Finally, the beginnings of a hybrid optical potential that is both phenomenological but contains microscopic elements can be explored further
and could help to control the extrapolation of these potentials away from stability. 
\\
\textbf{Summary}
 - We will develop a method to generate an effective nucleon-nucleon interaction in nuclei improve hadron-induced reaction calculations. The need for this effective interaction is
 most clearly demonstrated in the specific reaction description of $^{40}$Ca$(p,2p)^{39}$K (see Fig.~\ref{fig:eep_p2p}) revealing a 20\% discrepancy in what boundstate information
 can be extracted from this reaction. Employing our proposed effective interaction in this calculation will remedy this situation, thus enabling the community to reliably extract
 nuclear structure properties from these complex nuclear reactions. This will be the first time a nucleon-nucleon interaction is dressed specifically with finite-nucleus
 information. The effective interaction can improve many other reaction calculations, two of which we will explicity demonstrate during our project (deuteron scattering and optical
 potentials). Without accurate nuclear-reaction predictions, very little could be learned from nuclear scattering experiments. Thus, our improvements to nuclear-reaction
 descriptions help to harness the wealth of experimental data that will come from the DOE-funded facility FRIB.  Thus, our improvements to nuclear-reaction predictions are vital to
 further unravelling the mystery how protons and neutrons arrange themselves in nuclei.

%\bibliographystyle{apsrev4-1}
   \bibliographystyle{elsarticle-test}
   \bibliography{proposal}

   \end{document}



%Next need to detail the steps needed to develop this ab initio optical potential.
%Important to point out that the ingredients calculated along the way have their own
%merit. For starters, the dressed vertex function will be useful for p2p (d,p) with
%GREGORY reactions to help include information about the nucelus in the NN
%interaction - reference recent arxiv paper here. 2) Then discuss how the HF
%self-energy can be inserted into the current DOM to add ab initio elements to the
%DOM. This will not only provide insight into the tensor force and relevance of
%particular NN interactions, but also show us how the energy dependence should look
%beyond the HF self-energy. Then finally the full ladder-summed self-energy which
%should approximate high-energy scattering well.  Be sure to specify the exact system
%I plan on testing with -> 40Ca since I have a good DOM description, it is double
%magic, and spherical.




%In order to explore this approximation, I plan to implement these
%concepts using DOM single-particle propagators. While this will make the optical potential
%inherently depend on the parameters of the DOM, it can demonstrate if this procedure captures a wider range of nucleon-nuclus scattering energies than comparable methods.

%can describe nucleon-nucleus scattering over a wider range of energies than other methods. 
%expand the energy domain where bridge the energy gap in optical
%potentials.

%Ah, could mention that I am striving to improve optical potentials in general, so
%also adding to phenomenological potentails to make them more controllable. 

%\section{Wim}

%\textbf{C2.b\hspace{0.2cm} Successful treatment of harder 
%nucleon-nucleon interactions in medium-heavy nuclei, nuclear saturation, 
%and the \textit{ab initio} optical potential}. $-$
%Tremendous progress has been made in recent years in applying many-body 
%techniques to the study of finite nuclei using nucleon-nucleon and 
%three-nucleon interactions based on chiral perturbation theory.
%Some recent publications illustrating this progress can be found in 
%Refs.~\cite{Soma14,Hagen14,Epelbaum14}.
%Due to the choice of the momentum cut-off in the original chiral 
%interactions  the resulting coverage of momentum space is rather small 
%and not capable of generating enough high-momentum nucleons as required 
%by experimental observations at Jefferson Lab.
%A consequence of this softness of the basic interaction is the 
%overbinding of heavier nuclei~\cite{Soma14} and the accompanying 
%underestimate of the size of the nucleus and the related unsatisfactory 
%saturation properties in nuclear matter at the two-body 
%level~\cite{Carbone14}.
%In addition, some methods require further softening of the interaction 
%with renormalization procedures thereby inducing higher-body 
%interactions~\cite{Furnstahl13}.
%While there is uncertainty about the precise nature of the short-range 
%part of the $NN$
%interaction, there is now evidence from recent lattice QCD calculations
%that a strong repulsive short-range core emerges from
%first principles, particularly when the pion mass is reduced towards 
%more
%realistic values~\cite{Ishii07,Ishii12,Inoue15}.
%The presence of SRC is thus corroborated by QCD
%simulations and strongly suggests that \textit{ab initio} nuclear 
%many-body
%calculations should properly include their consequences.
%In the following we discuss several aspects of including a proper 
%treatment of SRC in calculations of finite nuclei, including the optical 
%potential, and the relation with the open question of nuclear-matter 
%saturation.
%\par\vspace{0.4ex}
%\par\noindent
%\textbf{C2.b1\hspace{0.2cm} Self-consistent determination of the nuclear 
%$G$-matrix in finite nuclei, the energy of the ground state, and elastic 
%nucleon scattering}. $-$
%As discussed in Ref.~\cite{Mahzoon14} the nonlocal DOM determination of 
%the sp propagator generates a binding of 7.91 MeV/A with a dominant 
%contribution of about 60--70\% from nucleons with momenta above 1.4 
%fm$^{-1}$ further emphasizing the important role of SRC in the binding 
%of nuclei.
%A large fraction of many-body calculations for finite nuclei that 
%explicitly
%deal with the strong short-range repulsion of the $NN$ interaction
%proceed by constructing a $G$-matrix effective interaction.
%This strategy is relevant for shell-model calculations~\cite{Brown01}, 
%no-core shell
%model techniques~\cite{Navratil00} (at the two-body cluster level),
%coupled-cluster approaches employing interactions with stronger 
%cores~\cite{Dean04}, and the FRPA implementation of the Green's function 
%method method~\cite{Dickhoff04}.
%Essentially all calculations employ intermediate states that only 
%propagate
%particles with kinetic energy while treating the Pauli operator at the 
%level
%of the chosen model space as accurately as possible~\cite{Hjorth94}.
%We note that very few serious attempts have been made to employ 
%$G$-matrices
%calculated for \textit{finite} nuclei for the construction of the 
%nucleon optical
%potential except for some of our efforts (see \textit{e.g.} 
%Ref.~\cite{Dussan11}) although the nuclear-matter route has been used 
%extensively for this purpose~\cite{Jeukenne76,Jeukenne77}.
%%Naturally, the nuclear-matter route can only yield information 
%concerning
%%the volume part of the optical potential, but this procedure has met
%%considerable success, although it has never been implemented at a truly
%%\textit{ab initio} level.

%During the current grant period we have confronted some of the technical 
%challenges to calculate
%the $G$-matrix employing the Green's function method in a finite nucleus 
%at positive energy. Initially started in collaboration with former 
%post-doctoral associate Dussan and subsequently with contributions from 
%graduate student Dong Ding~\cite{DingPhD} we propose to continue this 
%project in collaboration with a new post-doc and possibly a new graduate 
%student with some small contributions from Atkinson.
%Ultimately we plan to propagate fully dressed particles in the ladder 
%equation
%instead of noninteracting ones that only experience the Pauli
%principle.
%Full dressing means that the sp propagators are
%solutions of the Dyson equation describing scattering states at positive 
%energy.
%These particles therefore experience the nucleus just like elastically 
%scattering nucleons.
%Our current DOM propagators will therefore provide an immediate 
%assessment of the possibilities of this approach when used in the 
%construction of the effective interaction with obvious relevance for the 
%$(p,pN)$ project discussed in Sec.~\textbf{C2.a3}.
%Apart from being capable of treating SRC completely, the nucleons 
%generating the effective interaction will actually propagate in a finite 
%nucleus without being pawned off to nuclear matter in some local density 
%approximation.
%The resulting effective interaction, referred to as the dressed 
%$G$-matrix,
%must then be folded with a hole propagator to generate the imaginary 
%part of the
%self-energy above the Fermi energy.
%The hole propagator will be generated by also calculating
%the self-energy using the dressed $G$-matrix (initially in second order)
%with an intermediate dressed two-hole--one-particle 
%propagator~\cite{Dussan11}.
%Both particle and hole propagators will then be obtained by solving the 
%Dyson
%equation.
%The whole process is repeated until self-consistency has been achieved.

%Several ingredients of this ambitious program have been developed and 
%are now available.
%These include the solution of the Dyson equation below the Fermi 
%energy~\cite{Dickhoff10a}
%for a self-energy in momentum space for a given $\ell,j$ combination.
%We have also implemented a momentum-space solution of the Dyson equation 
%for positive energies
%in Ref.~\cite{Dussan11}.
%Our experience with the DOM and the CDBonn self-energy employed in 
%Ref.~\cite{Dussan11}
%has clarified that the conventional partial wave basis to calculate the 
%effective interaction is not viable since it will require a prohibitive 
%amount of recoupling of very large values of angular momentum.
%These are encountered when positive energy scattering amplitudes are 
%evaluated exhibiting very slow convergence in $\ell$~\cite{Dussan11}.
%The self-energy below the Fermi energy will be calculated as in 
%Refs.~\cite{Muther95,Dussan11}.
%At this stage the folding of the interaction with the one-body density 
%matrix through a momentum-space integral has been accomplished.
%The only remaining development is therefore the calculation of the 
%dressed $G$-matrix.

%We will employ a momentum vector representation which we have already 
%implemented for the free $NN$ $\mathcal{T}$-matrix that directly 
%generates the relevant scattering amplitudes and corresponding $NN$ 
%cross sections.
%Such a calculation was pioneered by the Groningen group using the 
%Dirac-Brueckner approach but has apparently not been applied 
%since~\cite{terHaar87}.
%We have constructed a propagator at positive energy in the sp momentum 
%and spin basis by summing for the relevant $\ell j$ partial wave 
%contributions from the momentum space DOM code that has been developed 
%for Ref.~\cite{Dussan11}.
%The relevant information is most efficiently contained in the reducible 
%(nondiagonal)
%self-energy ($\mathcal{T}$-matrix) which has a smooth energy and 
%momentum
%dependence (apart from possible low-energy resonances).
%Part of the self-consistency treatment requires the construction of the 
%momentum vector basis code to solve the sp scattering problem that 
%directly generates the relevant propagators.
%We are presently constructing the convolution
%of two such propagators to produce the intermediate
%propagator of the dressed $G$-matrix.

%The high level of technical and computational expertise required for 
%this project suggests that
%the participation of a post-doctoral associate is more appropriate than 
%even an accomplished graduate student.
%The \textit{ab initio} approach also provides the theoretical 
%underpinning of the
%proposed research of Sec.~\textbf{C2.a}.
%The project will also generate an
%\textit{ab initio} description of nucleon elastic scattering in a wide 
%energy domain.
%%We note that the construction of this medium-modified two-body 
%interaction at positive energy is ideally suited to facilitate future 
%analyses of the $(p,pN)$ reaction.

%We are currently adapting the nuclear matter techniques for data 
%handling of a self-consistent calculation using the experience employed 
%in the work described in Ref.~\cite{Ding:2016,DingPhD}.
%We further plan to employ a complex pole approximation to the sp 
%propagators that will map the problem precisely on a nuclear-matter-type 
%calculation.
%Since we have already implemented the calculation of the hole spectral 
%function we only need the convolution of the in-medium ladder-summed 
%interaction with this quantity to obtain a realistic \textit{ab initio} 
%optical potential for finite nuclei.
%Current implementation continues with the correlated HF contribution to 
%the self-energy also part of the proposed research to extend the reach 
%of the DOM potentials (see Sec.~\textbf{C2.a6}).
%%We are employing a code supplied by Morten Hjorth-Jensen that performs 
%a HF calculation with the N3LO chiral interaction.
%%We are using this code to test our calculations in momentum space for 
%this part of the self-energy.
%Initial calculations will thus explore the $\mathcal{T} \rho$ 
%approximation to the multiple scattering formulation of the optical 
%potential~\cite{Weppner98,Weppner06,Weppner12} but with a sophisticated 
%DOM one-body density matrix and will be subsequently improved by 
%incorporating the ingredients identified above.\\
%\textit{(post-doctoral associate, Atkinson, Rios (Surrey), Polls 
%(Barcelona), Dickhoff)}
